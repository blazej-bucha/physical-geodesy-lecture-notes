% =============================================================================
%
% This is the LaTeX source code of the lecture notes.
%
%
% Author:   Blazej Bucha
% Year:     2022
% Contact:  blazej.bucha@stuba.sk
% Encoding: UTF-8
%
% =============================================================================






% LaTeX Packages
% =============================================================================

\documentclass[a4paper, 12pt]{book}
\usepackage[slovak]{babel}
\usepackage[utf8]{inputenc}
\usepackage[round]{natbib}
\usepackage{amsmath}
\usepackage{upgreek}
\usepackage{listings}
\usepackage{xcolor}
\usepackage{graphicx}
\usepackage{accsupp}
\usepackage[top=2.5cm, bottom=2.5cm, left=2.5cm, right=2.5cm]{geometry}
\linespread{1.2}

% =============================================================================






% Define and set the style of code listings
% =============================================================================

% Define custom colours
\definecolor{comments}{rgb}{0.65, 0.65, 0.65}
\definecolor{strings}{rgb}{0.0, 0.3, 0.7}
\definecolor{linenumbers}{rgb}{0.5, 0.5, 0.5}
\definecolor{keywords}{rgb}{0.85, 0.0, 0.0}
\definecolor{backcolour}{rgb}{0.98, 0.98, 0.98}


% Define custom listing style
\lstdefinestyle{codestyle}
{
    backgroundcolor=\color{backcolour},
    commentstyle=\color{comments},
    keywordstyle=\color{keywords},
    numberstyle=\tiny\noncopynumber,
    columns=flexible,
    stringstyle=\color{strings},
    basicstyle=\ttfamily\footnotesize,
    breakatwhitespace=false,
    breaklines=true,
    captionpos=b,
    keepspaces=true,
    numbers=left,
    numbersep=5pt,
    showspaces=false,
    showstringspaces=false,
    showtabs=false,
    tabsize=2
}


% Defines a new command that ensures we can copy the source code from the 
% compiled PDF without copying the line numbers
\newcommand{\noncopynumber}[1]
{
    \BeginAccSupp{method=escape,ActualText={}}
    #1
    \EndAccSupp{}
}


% Apply the custom listing style
\lstset{style=codestyle}


% It seems the "listings" package is not able to encode the nice special Slovak 
% characters such as "á", "ľ", etc.  Next follows a brute force solution.
\lstset{
  literate={á}{{\'a}}1
           {ä}{{\" a}}1
           {č}{{\v c}}1
           {ď}{{d\kern-0.07cm\char39\kern-0.07cm}}1
           {é}{{\'e}}1
           {í}{{\'i}}1
           {ĺ}{{\'l}}1
           {ľ}{{l\kern-0.12cm\char39\kern-0.05cm}}1
           {ň}{{\v n}}1
           {ó}{{\'o}}1
           {ô}{{\^o}}1
           {ŕ}{{\'r}}1
           {š}{{\v s}}1
           {ť}{{t\kern-0.10cm\char39\kern-0.05cm}}1
           {ú}{{\'u}}1
           {ý}{{\'y}}1
           {ž}{{\v z}}1
           {Á}{{\'A}}1
           {Ä}{{\" A}}1
           {Č}{{\v C}}1
           {Ď}{{\v D}}1
           {É}{{\'E}}1
           {Í}{{\'I}}1
           {Ĺ}{{\'L}}1
           {Ľ}{{L\kern-0.12cm\char39\kern-0.00cm}}1
           {Ň}{{\v N}}1
           {Ó}{{\'O}}1
           {Ô}{{\^O}}1
           {Ŕ}{{\'R}}1
           {Š}{{\v S}}1
           {Ť}{{\v T}}1
           {Ú}{{\'U}}1
           {Ý}{{\'Y}}1
           {Ž}{{\v Z}}1
}


% Add the "as" keyword to the Python listings
\lstdefinelanguage{mypython}[]{Python}{morekeywords={as,True,False}}


% Caption title for listings
\renewcommand\lstlistingname{Zdrojový kód}

% =============================================================================






% Custom commands
% =============================================================================
\newcommand{\diff}{\mathrm d}
\newcommand{\grad}{\mathrm{grad}}
\newcommand{\gidx}{\mathrm g}
\let\vec\mathbf

% Let's change the "Dodatok" label of chapters after Conclusions to "Príloha"
\addto\captionsslovak{\renewcommand\appendixname{Príloha}}

% =============================================================================






% =============================================================================

\sloppy
\begin{document}

% -----------------------------------------------------------------------------
\tableofcontents
\newpage
% -----------------------------------------------------------------------------






% -----------------------------------------------------------------------------

\chapter*{Úvod}






% -----------------------------------------------------------------------------

\chapter{Fyzikálna geodézia}

Tvar Zeme možno definovať niekoľkými spôsobmi v závislosti od kontextu 
\citep{MoritzTheFigureOfTheEarth}.  Azda najprirodzenejšie je definovať tvar 
Zeme jej skutočným fyzickým povrchom, teda tou časťou Zeme, ktorá je viditeľná 
voľným okom.  Takýto povrch by však obsahoval aj nespočetné množstvo zložitých 
terénnych útvarov, napríklad náhle skokovité zmeny terénu.  Hoci sú takéto 
oblasti krásne na pohľad, znemožňujú aplikovať časť potrebného matematického 
aparátu, napríklad plošnú integráciu.  Skúmať takto chápaný zemský povrch je 
preto možné iba po istom vyhladení.  Napriek tomu, ide o veľmi komplexnú 
plochu.

Pri definovaní tvaru Zeme však môžeme vziať v úvahu skutočnosť, že približne 
dve tretiny zemského povrchu tvorí voda v podobe oceánov a morí.  Ak by sme 
s podobnou mierou hladkosti dokázali predĺžiť povrch oceánov a morí aj popod, 
resp. ponad pevnú časť zemského povrchu, získali by sme v porovnaní s fyzickým 
povrchom Zeme značne jednoduchší a hladší, no pritom stále verný tvar Zeme.  
Voda má za istých okolností navyše tú výhodnú vlastnosť, že existuje fyzikálna 
veličina, ktorá je na jej povrchu konštantná.  Táto vlastnosť by teda následne 
mohla umožniť akúsi predikciu morskej hladiny aj popod, resp. ponad kontinenty.  
Inými slovami, v tejto definícií tvaru Zeme by išlo o geometricky jednoduchšiu 
plochu, navyše tentokrát už aj s istými relatívne jednoducho popísateľnými 
fyzikálnymi vlastnosťami.  Takto definovaný tvar Zeme bol navrhnutý 
C. F. Gaussom (1777--1855) a podľa návrhu J. B. Listinga (1808--1882) ho 
nazývame \emph{geoid}.  Veličina, ktorá je konštantná na povrchu geoidu sa 
nazýva \emph{tiažový potenciál}.

\emph{Fyzikálna geodézia} je vedná disciplína zaoberajúca sa určovaním tvaru 
Zeme, jej tiažovým poľom a ich zmenami v čase.  Študované sú oba vyššie opísané 
prístupy k tvaru Zeme, pričom historicky bol predmetom záujmu ako prvý geoid.  
Hoci prirodzená, no matematicky odvážna myšlienka určovať priamo fyzický povrch 
Zeme pochádza už od E. H. Brunsa (1848--1919), do popredia záujmu ju dostal až 
v polovici 20. storočia M. S. Molodenskij (1909--1991).  Dôležitá je však 
skutočnosť, že i v tomto prípade je tvar Zeme určovaný z informácie o tiažovom 
poli.  Študovať tvar Zeme, čo je jedna zo základných úloh geodézie, teda 
znamená študovať aj tiažové pole Zeme.

\section{Newtonov gravitačný zákon}
\label{sec:newton_law}

%Uvažujme hmotný bod o hmotnosti $m$.  V zmysle gravitačného zákona 
%formulovaného I. Newtonom (1642--1727) tento hmotný bod generuje gravitačný 
%potenciál $V$.  Vo vzdialenosti $l$ od tohto hmotného je gravitačný potenciál 
%daný vzťahom
%%
%\begin{equation}
%V = G \frac{m}{l}{,}
%\end{equation}
%%
%kde $G = 6.6742 \times 10^{-11} \ \mathrm{m}^3 \ \mathrm{kg}^{-1} 
%\ \mathrm{s}^{-2}$ je Newtonova gravitačná konštanta.

Dva hmotné body $P$ a $Q$ nachádzajúce sa od seba vo vzdialenosti $l$ sa 
navzájom priťahujú \emph{gravitačnou silou}
%
\begin{equation}
\label{eq:newton_law}
\vec F = -G \frac{m_P \, m_Q}{l^2} \, \frac{\vec r}{l}{.}
\end{equation}
%
Symbol $G$ označuje gravitačnú konštantu, $m_P$ a $m_Q$ predstavujú hmotnosti 
hmotných bodov a $\vec r$ je vektor definovaný bodmi $P$ a $Q$ so začiatkom 
v~bode $Q$.  Záporné znamienko v rovnici znamená, že vektor sily má začiatok 
v bode $P$ a smeruje do bodu $Q$ (Obrázok~\ref{fig:newton_law}).

\begin{figure}[b]
\centering
\input{./fig-newton-law.pdf_tex}
\caption{Gravitačná sila $\vec F$ pôsobiaca medzi hmotnými bodmi $P$ a $Q$} 
\label{fig:newton_law}
\end{figure}

Rovnica~(\ref{eq:newton_law}) predstavuje \emph{gravitačný zákon} 
publikovaný I. Newtonom (1642--1727) v práci \emph{Philosophi\ae Naturalis 
Principia Mathematica} z roku 1687.  Hovorí, že každý hmotný bod vo vesmíre 
priťahuje každý iný hmotný bod silou, ktorej smer je daný spojnicou týchto 
bodov a veľkosť je priamo úmerná súčinu ich hmotností a nepriamo úmerná štvorcu 
ich vzájomnej vzdialenosti \citep{Kellogg1967}.  \emph{Hmotný bod} je 
idealizovaný pojem predstavujúci časticu s nulovým rozmerom a konečnou 
nenulovou hmotnosťou bez akýchkoľvek ďalších vlastností či štruktúry.  Hodnota 
gravitačnej konštanty odporúčaná od roku 2018 \emph{Komisiou pre dáta vo vede 
a technológiách} (CODATA) je
%
\begin{equation}
G = (6.67430 \pm 0.00015) \times 10^{-11} \ \mathrm{m}^3 \ \mathrm{kg}^{-1} 
\ \mathrm{s}^{-2}{.}
\end{equation}

Newtonov gravitačný zákon vysvetľuje mnoho prírodných javov.  Popisuje 
mechanizmus pohybu planét Slnečnej sústavy, prílivu a odlivu, či odpovedá na 
otázku, prečo majú planéty približne sférický tvar.  Je pozoruhodné, že hoci 
Newtonov gravitačný zákon bol odpozorovaný z pohybu nebeských telies, zhoduje 
sa i s experimentmi, ktoré boli vykonané na vzdialenosť niekoľkých desiatok 
mikrometrov \citep{Lee2020}.

Hoci rovnica~(\ref{eq:newton_law}) vysvetľuje množstvo javov, je azda 
potrebné spomenúť, že Newtonov gravitačný zákon je v skutočnosti 
\emph{nesprávny} \citep{Feynman}.  Predpokladá napríklad, že zmena hmotnosti či 
vzdialenosti má okamžitý gravitačný účinok, teda i v prípade telies 
nachádzajúcich sa vo veľkých vzdialenostiach.  Tento predpoklad je však 
v rozpore s experimentálne potvrdenou Einsteinovou špeciálnou teóriou 
relativity.  Tá hovorí, že žiadny signál, a teda ani ten gravitačný, sa nemôže 
šíriť rýchlosťou väčšou ako je rýchlosť svetla.  Pre účely fyzikálnej geodézie 
sú však chyby spôsobené nepresnosťami Newtonovho gravitačného zákona väčšinou 
zanedbateľné, a tak táto teória, aj dnes, viac ako tristo rokov od jej vzniku, 
tvorí chrbtovú kosť fyzikálnej geodézie.

\subsection{Gravitačné zrýchlenie}
\label{sec:gg}

Gravitačná sila~(\ref{eq:newton_law}) závisí od hmotnosti oboch hmotných 
bodov medzi ktorými pôsobí.  Skúsme preto normovať gravitačnú silu hmotnosťou 
hmotného bodu $P$, prípadne aplikovať podmienku $m_P = 1\ \mathrm{kg}$.  Oba 
spôsoby vedú ekvivalentne k veličine, ktorá sa nazýva \emph{gravitačné 
zrýchlenie},
%
\begin{equation}
\label{eq:gg_point_mass}
\vec g_\gidx(P) = \frac{\vec F}{m_P} = -G \frac{m}{l^2} \, \frac{\vec r}{l}{,}
\end{equation}
%
kde $m = m_Q$.  Rovnica~(\ref{eq:gg_point_mass}) hovorí, že hmotný 
bod~$Q$ udeľuje hmotnému bodu $P$ zrýchlenie $\vec g_\gidx(P)$.   Inými 
slovami, gravitačné zrýchlenie $\vec g_\gidx(P)$ nezávisí od hmotnosti hmotného 
bodu $P$, začiatok má v bode $P$ a smeruje do hmotného bodu $Q$.  Skutočnosť, 
že gravitačné zrýchlenie priťahovaného telesa nezávisí od jeho hmotnosti 
experimentálne potvrdil Gaileo Gailei (1564--1642).  V súvislosti s hmotným 
bodom $P$ je preto možné pre jednoduchosť vynechať prívlastok \emph{hmotný} 
a považovať $P$ za geometrický bod bez fyzikálnych vlastností.

Z predchádzajúcej kapitoly vyplýva, že v sústave $N$ hmotných bodov je potrebné 
na získanie celkového gravitačného zrýchlenia započítať všetky čiastkové 
zrýchlenia udeľované jednotlivými hmotnými bodmi 
(Obrázok~\ref{fig:gg_n_point_masses}).  Gravitačné zrýchlenie je v takom 
prípade dané vektorovým súčtom,
%
\begin{equation}
\label{eq:gg_N_point_masses}
\vec g_\gidx(P) = \sum_{i = 1}^{N}\vec g_{\gidx,i}(P) = -G \sum_{i = 1}^{N} 
\frac{m_i}{l_i^2} \, \frac{\vec r_i}{l_i}{.}
\end{equation}

\begin{figure}[b]
\centering
\input{./fig-gg-n-point-masses.pdf_tex}
\caption{Vektory gravitačného zrýchlenia v bode $P$ generované sústavou piatich 
hmotných bodov ***CENTROVAT OBRAZOK***}
\label{fig:gg_n_point_masses}
\end{figure}

Newtonov gravitačný zákon~(\ref{eq:newton_law}) 
a rovnice~(\ref{eq:gg_point_mass}) 
a (\ref{eq:gg_N_point_masses}) platia iba pre hmotné body.  
Tento tvar je možné aplikovať napríklad na popis pohybu planét Slnečnej 
sústavy.  Rozmery planét sú totiž natoľko zanedbateľné v porovnaní s ich 
vzájomnou vzdialenosťou, že sa nedopustíme nerozumne veľkej chyby, ak hmotnosti 
týchto telies sústredíme do ich ťažísk a zanedbáme ich tvar.  Menej priaznivá 
situácia nastáva vtedy, keď sa bod $P$ nachádza v blízkosti priťahujúceho 
telesa, ktoré má komplikovaný tvar a hustotu.  Príkladom je družica na nízkej 
obežnej dráhe Zeme.  V súlade s predchádzajúcim odsekom skúsme preto nahradiť 
Zem množinou hmotných bodov, či už s konečným alebo nekonečným počtom, 
a získajme výsledné gravitačné zrýchlenie pôsobiace na družicu 
vzťahom~(\ref{eq:gg_N_point_masses}).  Označenie takéhoto 
riešenia za nesprávne nebude prekvapivé, keď si spomenieme 
(Kapitola~\ref{sec:newton_law}), že hmotný bod bol definovaný 
ako bezrozmerný.  Bezrozmernými hmotnými bodmi nemožno aproximovať objem 
telesa.

Zaveďme preto pojem \emph{všeobecné teleso}, ktoré má konečný objem $\tau$ 
a ohraničenú hustotu $\rho$ (Obrázok~\ref{fig:gravitating_body}).  Rozdeľme 
toto teleso na nekonečne veľa diferenciálnych hmotných elementov $\diff 
m = \rho \, \diff \tau$.  Hustota $\rho$ sa môže meniť medzi hmotnými elementmi 
$\diff m$ v závislosti od štruktúry daného telesa, ale nesmie sa meniť v rámci 
elementu samotného.  Aby sme mohli aplikovať Newtonov gravitačný zákon, ktorý 
je formulovaný pre hmotné body, budeme musieť predpokladať, že diferenciálne 
hmotné elementy $\diff m$, akokoľvek sú malé, nikdy nie sú identické s hmotnými 
bodmi; navyše, hmota týchto elementov je koncentrovaná do jedného bodu 
\citep{Kellogg1967}.

\begin{figure}
\centering
\input{./fig-gravitating-body.pdf_tex}
\caption{Všeobecné teleso}
\label{fig:gravitating_body}
\end{figure}

Po uvážení $\diff m = \rho \, \diff \tau$ bude gravitačné zrýchlenie generované 
hmotným elementom $\diff m$ dané vzťahom
%
\begin{equation}
\diff \vec g_\gidx(P) = -G \frac{\vec r}{l^3} \rho \, \diff\tau{,}
\end{equation}
%
kde vektor $\vec r$ je definovaný rozdielom polohových vektorov bodov $P$ 
a $Q$,
%
\begin{equation}
\label{eq:r}
\vec r = 
%
\begin{bmatrix}
x - x_Q \\
y - y_Q \\
z - z_Q
\end{bmatrix}
{,}
\end{equation}
%
a $l$ je Euklidovská vzdialenosť medzi týmito bodmi,
%
\begin{equation}
\label{eq:l}
l = | \vec r | = \sqrt{(x - x_Q)^2 + (y - y_Q)^2 + (z - z_Q)^2}{.}
\end{equation}
%
Celkové gravitačné zrýchlenie generované všeobecným telesom je teda
%
\begin{equation}
\label{eq:gg_body}
\vec g_\gidx(P) = -G \iiint_{\tau} \frac{\vec r}{l^3} \, \rho \, \diff\tau{.}
\end{equation}
%
Tento vzťah už môže byť korektne použitý na výpočet gravitačného zrýchlenia, 
ktoré udeľuje všeobecné teleso inému telesu s ťažiskom v bode $P$ (napríklad 
Zem družici).

Vektor gravitačného zrýchlenia veľmi úzko súvisí s veličinou, ktorú je možné 
odmerať na povrchu Zeme, a tak má pre fyzikálnu geodéziu kľúčový význam.  
V sústave jednotiek SI má gravitačné zrýchlenie jednotku $\mathrm{m}\ 
\mathrm{s}^{-2}$.  Veľmi často je možné stretnúť sa aj s jednotkou $1\ 
\mathrm{Gal} = 10^{-2}\ \mathrm{m}\ \mathrm{s}^{-2}$ pomenovanou podľa Galilea 
Galileiho, prípadne s jej násobkami $1\ \mathrm{mGal} = 10^{-5}\ \mathrm{m}\ 
\mathrm{s}^{-2}$ a $1\ \upmu \mathrm{Gal} = 10^{-8}\ \mathrm{m}\ 
\mathrm{s}^{-2}$.

\subsection{Gravitačný potenciál}
\label{sec:vg}

Gravitačné zrýchlenie $\vec g_\gidx$ je vektorová veličina, ktorá má v každom 
bode definovanú trojicu reálnych čísel,
%
\begin{equation}
\label{eq:gg_elements}
\vec g_{\gidx}(P) =
\begin{bmatrix}
g_{\gidx,x}(P) \\[2ex]
g_{\gidx,y}(P) \\[2ex]
g_{\gidx,z}(P)
\end{bmatrix}
{.}
\end{equation}
%
V niektorých situáciach je výhodnejšie pracovať so skalárnou veličinou, teda 
takou, ktorá má v bode~$P$ definované práve jedno reálne číslo.  Tento prístup 
je mladší ako Newtonov gravitačný zákon zhruba o jedno storočie 
\citep{MacMillan1930,Jekeli2015} a fenomén gravitácie chápe ako \emph{pole}, 
ktorého vlastnosti môžu byť popísané rôznymi vzájomne súvisiacimi veličinami, 
pričom však ide stále o to isté pole.  S konceptom potenciálu poľa prišiel 
podľa \cite{MacMillan1930} taliansky matematik a astronóm J. L. Lagrange 
(1736--1813).  Ak by sme teda našli skalárnu veličinu gravitačného poľa, 
informáciu poskytnutú troma číslami v podobe gravitačného vektora 
(vzťah~\ref{eq:gg_elements}) by sme dokázali zredukovať na jedno jediné číslo 
bez straty informácie o poli samotnom.  Skalárna veličina gravitačného poľa sa 
nazýva \emph{gravitačný potenciál} a označuje sa symbolom $V_\gidx$.

Gravitačný potenciál definujeme pomocou gravitačného zrýchlenia, ktoré, ako sme 
videli v predchádzajúcej časti, je definované pomocou gravitačnej sily 
vyplývajúcej z Newtonovho gravitačného zákona.  Pre lepšie pochopenie vzťahu 
medzi gravitačným zrýchlením a gravitačným potenciálom však najprv predstavíme 
diferenciálny operátor gradient a až následne pristúpime k samotnej definícii.

V trojrozmernom pravouhlom súradnicovom systéme $x$, $y$, $z$ je operátor 
gradient definovaný nasledovne,
%
\begin{equation}
\label{eq:gradient}
\nabla = \grad = \vec e_1 \frac{\partial}{\partial x} + \vec e_2 
\frac{\partial}{\partial y} + \vec e_3 \frac{\partial}{\partial z} =
\begin{bmatrix}
\dfrac{\partial}{\partial x} \\[2ex]
\dfrac{\partial}{\partial y} \\[2ex]
\dfrac{\partial}{\partial z}
\end{bmatrix}
{,}
\end{equation}
%
kde $\vec e_1$, $\vec e_2$ a $\vec e_3$ predstavujú jednotkové vektory 
rovnobežné so smerom súradnicových osí $x$, $y$ a $z$.  Aplikáciou operátora 
gradient na \emph{skalárnu} funkciu $f(x, y, z)$ získame \emph{vektorovú} 
funkciu,
%
\begin{equation}
\begin{split}
\vec f(x, y, z) &= \nabla f(x, y, z) = \grad \ f(x, y, z)\\
%
&= \vec e_1 \frac{\partial f(x, y, z)}{\partial x} + \vec e_2 \frac{\partial 
f(x, y, z)}{\partial y} + \vec e_3 \frac{\partial f(x, y, z)}{\partial z} =
\begin{bmatrix}
\dfrac{\partial f(x, y, z)}{\partial x} \\[2ex]
\dfrac{\partial f(x, y, z)}{\partial y} \\[2ex]
\dfrac{\partial f(x, y, z)}{\partial z}
\end{bmatrix}
{.}
\end{split}
\end{equation}
%
Vektorová funkcia $\vec f$ \emph{udáva smer a veľkosť najväčšieho nárastu} 
funkcie $f$ v bode so súradnicami $x$, $y$ a $z$.

Operátor gradient hrá v teórii fyzikálnych polí dôležitú úlohu, pretože 
umožňuje získať informácie o lokálnych charakteristikách poľa.  Medzi tieto 
informácie patrí smer a veľkosť najväčšieho nárastu funkcie, ale aj popis 
rýchlosti jej zmeny v smere súradnicových osí.\footnote{Rýchlosť zmeny funkcie 
je tu myslená v priestorovom zmysle, nie časovom.}  Vhodnou aplikáciou 
rotačných matíc je možné následne získať informáciu o rýchlosti zmeny funkcie 
v ľubovoľnom smere, čo je v mnohých situáciach veľmi nápomocné.  Ak si 
predstavíme povrch Zeme ako dvojrozmernú funkciu zemepisnej šírky a dĺžky, 
potom aplikáciou operátora gradient na túto funkciu získame (nielen) 
v kopcovitom teréne informáciu o smere, ktorým sa z tohto bodu zrejme nechceme 
vydať na bicykli.  Ak však dáme pred vektorovú funkciu znamienko mínus, zvyšok 
práce môžeme nechať na gravitačné pole.  Numerická ukážka aplikácie operátora 
gradient je spolu so zdrojovým kódom uvedená 
v Prílohe~\ref{app:numerical_application_of_gradient}.

Po tejto príprave môžeme definovať gravitačný potenciál nasledovne.  Gravitačný 
potenciál je skalárna funkcia, ktorá vyhovuje rovnici 
\citep{SansoGeoidDetermination}
%
\begin{equation}
\label{eq:gg_grad_vg}
\vec g_\gidx(P) = \nabla V_\gidx(P)
\end{equation}
%
a spĺňa podmienku
%
\begin{equation}
\label{eq:vg_at_infty}
\lim_{P \to \infty} V_\gidx(P) = 0{.}
\end{equation}
%
Vzťah~(\ref{eq:gg_grad_vg}) hovorí, že vektor gravitačného zrýchlenia udáva 
smer a veľkosť najväčšieho nárastu gravitačného potenciálu.  
Rovnica~(\ref{eq:vg_at_infty}) znamená, že gravitačný potenciál nadobúda 
v nekonečnej vzdialenosti od telesa nulovú hodnotu a zabezpečuje, že existuje 
práve jedna funkcia $V_\gidx$ vyhovujúca rovnici~(\ref{eq:gg_grad_vg}).  Keďže 
podmienka (\ref{eq:vg_at_infty}) bola definovaná (zvolená) dodatočne, 
\emph{gravitačný potenciál je relatívna veličina}.

Gravitačný potenciál hmotného bodu je daný nasledovne,
%
\begin{equation}
\label{eq:vg_point_mass}
V_\gidx(P) = \frac{G \, m}{l}{.}
\end{equation}
%
Overme, či tento vzťah vyhovuje rovniciam~(\ref{eq:gg_grad_vg}) 
a (\ref{eq:vg_at_infty}).  Aplikovaním operátora gradient~(\ref{eq:gradient}) 
na gravitačný potenciál~(\ref{eq:vg_point_mass}),
%
\begin{equation}
\label{eq:gg_from_vg_point_mass}
\nabla V_\gidx(P) = G \, m \, \nabla \left( \frac{1}{l} \right) =
%
G \, m
\begin{bmatrix}
\dfrac{\partial}{\partial x} \left( \dfrac{1}{l} \right)\\[2ex]
\dfrac{\partial}{\partial y} \left( \dfrac{1}{l} \right)\\[2ex]
\dfrac{\partial}{\partial z} \left( \dfrac{1}{l} \right)
\end{bmatrix}
%
=
%
-G \, m
%
\begin{bmatrix}
\dfrac{x - x_Q}{l^3}{,}\\[2ex]
\dfrac{y - y_Q}{l^3}{,}\\[2ex]
\dfrac{z - z_Q}{l^3}
\end{bmatrix}
{,}
\end{equation}
%
získame vektor gravitačného zrýchlenia $\vec g_\gidx(P)$ 
z rovnice~(\ref{eq:gg_point_mass}), čím bola dokázaná 
rovnosť~(\ref{eq:gg_grad_vg}).  Platnosť vzťahu~(\ref{eq:vg_at_infty}), teda
%
\begin{equation}
\lim_{l \to \infty} \frac{G \, m}{l} = 0{,}
\end{equation}
%
je zrejmá.

V sústave $N$ hmotných bodov je celkový gravitačný potenciál daný súčtom 
čiastkových príspevkov v dôsledku jednotlivých hmotných bodov,
%
\begin{equation}
\label{eq:vg_N_point_masses}
V_\gidx(P) = \sum_{i = 1}^{N} V_{\gidx,i}(P) = G \sum_{i = 1}^{N}\frac{
m_i}{l_i}{.}
\end{equation}
%
Obdobným spôsobom ako v rovnici~(\ref{eq:gg_from_vg_point_mass}) je možné 
presvedčiť sa, že aplikovaním operátora gradient na gravitačný 
potenciál~(\ref{eq:vg_N_point_masses}) získame vektor gravitačného 
zrýchlenia~(\ref{eq:gg_N_point_masses}).  Teda i v prípade gravitačného 
potenciálu generovaného sústavou $N$ hmotných bodov sú 
rovnice~(\ref{eq:gg_grad_vg}) a (\ref{eq:vg_at_infty}) splnené.

Použitím podobnej úvahy ako v Kapitole~\ref{sec:gg} zistíme, že gravitačný 
potenciál všeobecného telesa je daný vzťahom
%
\begin{equation}
\label{eq:vg_body}
V_\gidx(P) = G \iiint_{\tau} \frac{\rho}{l} \diff\tau{.}
\end{equation}
%
Dôkaz, že i rovnica~(\ref{eq:vg_body}) vyhovuje prijatej definícii gravitačného 
potenciálu (\ref{eq:gg_grad_vg} a \ref{eq:vg_at_infty}) je možné nájsť 
napríklad v \cite{MacMillan1930}.

K interpretácii gravitačného potenciálu je možné pristúpiť aj z fyzikálneho 
hľadiska.  V tejto práci sa obmedzíme iba na tvrdenie, že hodnota gravitačného 
potenciálu predstavuje prácu, ktorú musí vykonať gravitačné pole pri 
premiestnení hmotného bodu s hmotnosťou $1\ \mathrm{kg}$ z miesta s nulovým 
potenciálom (v geodézii nekonečno, pozri vzťah~\ref{eq:vg_at_infty}) do bodu $P$ 
\citep{MacMillan1930,Kellogg1967,TorgeGeodesy}.

Na rozdiel od vektora gravitačného zrýchlenia, gravitačný potenciál nedokážeme 
priamo merať.  Vo fyzikálnej geodézii vystupuje často ako neznáma funkcia, 
ktorú sa snažíme určiť, napríklad z vektora gravitačného zrýchlenia.  V sústave 
SI má gravitačný potenciál fyzikálnu jednotku $\mathrm{m}^2\ \mathrm{s}^{-2}$.






% -----------------------------------------------------------------------------

\section{Teória potenciálu}

Rovnica~(\ref{eq:vg_body}) sa v literatúre zvykne nazývať \emph{Newtonov 
integrál}.  Hovorí, že ak poznáme tvar a priestorové rozloženie hustoty telesa, 
gravitačný potenciál generovaný týmto telesom môže byť vypočítaný.  Newtonov 
integrál má kľúčový význam pre fyzikálnu geodéziu, nakoľko v gravitačnom 
potenciáli je obsiahnutá informácia o celom gravitačnom poli (pozri 
Kapitolu~\ref{sec:newton_law}).  Zo vzťahu~(\ref{eq:vg_body}) je možné odvodiť 
množstvo ďalších veličín gravitačného poľa, okrem iného napríklad i vektor 
gravitačného zrýchlenia~(\ref{eq:gg_body}) (vzťahy~\ref{eq:vg_body} 
a \ref{eq:l} dosadíme do \ref{eq:gg_grad_vg}).

Newtonov integrál položil základy novej oblasti matematiky a matematickej 
fyziky s názvom \emph{teória potenciálu}.  Podľa \cite{MacMillan1930} siahajú 
jej začiatky k francúzskemu matematikovi, fyzikovi, astronómovi a politikovi 
P. S. Laplaceovi (1749--1827).  Ten si uvedomil, že gravitačný potenciál 
každého všeobecného telesa (v zmysle definície z Kapitoly~\ref{sec:gg}) spĺňa 
určité podmienky, čím vzniká zaujímavá množina funkcií hodná podrobnejšieho 
štúdia.  Významnosť teórie potenciálu iba potvrdzuje skutočnosť známa približne 
od čias C. F. Gaussa, a síce že túto teóriu je možné aplikovať nielen na 
gravitačné pole, ale  napríklad aj na magnetické, či elektrostatické pole 
(spomeňme si na Coulombov zákon a porovnajme ho s Newtonovým gravitačným 
zákonom~\ref{eq:newton_law}).  Teória potenciálu teda študuje 
vzťah~(\ref{eq:vg_body}) vo všeobecnom zmysle, pričom sa zaoberá rozličnými 
zdrojmi poľa (hmotný bod, hmotná priamka, všeobecné teleso a pod.) s rôznym 
priestorovým rozložením hustôt (konštantná, premenlivá, kladná, dokonca 
i záporna a pod.) a neobmedzuje sa iba na gravitačné pole.  Hoci môže azda táto 
úloha pôsobiť jednoducho, v skutočnosti je nesmierne náročná.  Čo sa napríklad 
stane v prípade, ak sa v rovnici~(\ref{eq:vg_body}) nachádza výpočtový bod $P$ 
vo vnútri telesa?  Za takýchto okolností musí nevyhnutne dôjsť k tomu, že 
súradnice bodu $P$, v ktorom hľadáme gravitačný potenciál, budú identické so 
súradnicami jedného hmotného elementu $\diff m$, teda ich vzájomná vzdialenosť 
vystupujúca v menovateli bude $l = 0$.  Konverguje alebo diverguje v takejto 
situácii integrál~(\ref{eq:vg_body})?  Jednou z mnohých ďalších výziev sú 
gravitačné polia generované objektmi komplikovaného tvaru, napríklad s náhlymi 
ostrými zmenami tvaru ako je tomu v prípade zemského povrchu.  Je preto 
prirodzené, že tieto matematické výzvy pritiahli záujem brilantných matematikov 
akými boli, okrem iných, J. L. Lagrange, P. S. Laplace, britský samouk G. Green 
(1793--1841), či C. F. Gauss.

\subsection{Laplaceova rovnica}

P. S. Laplace zistil, že gravitačný potenciál každého všeobecného telesa spĺňa 
v bode $P(x, y, z)$ \emph{mimo tohto telesa} rovnicu
%
\begin{equation}
\label{eq:vg_laplace_cart}
\nabla^2 V_\gidx = \Delta V_\gidx = \frac{\partial^2 V_\gidx}{\partial x^2}
+ \frac{\partial^2 V_\gidx}{\partial y^2} + \frac{\partial^2 V_\gidx}{\partial 
z^2} = 0{.}
\end{equation}
%
Operátor $\nabla^2 = \Delta$ sa nazýva \emph{Laplaceov operátor} 
a rovnica~(\ref{eq:vg_laplace_cart}) na nazýva \emph{Laplaceova rovnica} pre 
gravitačný potenciál.  Jednou z dôležitých vlastností Laplaceovho operátors je 
tá, že \emph{lineárny}, teda pre funkcie $f$ a $g$ vyhovujúce 
rovnici~(\ref{eq:vg_laplace_cart}) a konštantu $c$ platí
%
\begin{equation}
\label{eq:laplace_additivity}
\nabla^2 \left(f + g \right) = \nabla^2 f + \nabla^2 g
\end{equation}
%
a
%
\begin{equation}
\label{eq:laplace_homogenity}
\nabla^2 (c \, f) = c \, \nabla^2 f{.}
\end{equation}

Skúsme overiť, či gravitačný potenciál hmotného bodu~(\ref{eq:vg_point_mass}) 
vyhovuje Laplaceovej rovnici.  Budeme uvažovať, že $l > 0$, nakoľko gravitačný 
potenciál hmotného bodu nie je definovaný pre $l = 0$.  
S uvážením~(\ref{eq:laplace_homogenity}), postačuje dokázať, že
%
\begin{equation}
\nabla^2 \left( \frac{1}{l} \right) = 0{,}
\end{equation}
%
pričom $c = G \, m$.  Vypočítajme teda druhé parciálne derivácie funkcie $1 
\slash l$,
%
\begin{equation}
\label{eq:l_2nd_derivatives}
\begin{split}
\frac{\partial^2}{\partial x^2} \left( \frac{1}{l} \right) &= 
-\frac{\partial}{\partial x} \left( \frac{x - x_Q}{l^3} \right) = \left(3 
\frac{(x - x_Q)^2}{l^5} - \frac{1}{l^3} \right){,}\\
%
\frac{\partial^2}{\partial y^2} \left( \frac{1}{l} \right) &= 
-\frac{\partial}{\partial y} \left( \frac{y - y_Q}{l^3} \right) = \left(3 
\frac{(y - y_Q)^2}{l^5} - \frac{1}{l^3} \right){,}\\
%
\frac{\partial^2}{\partial z^2} \left( \frac{1}{l} \right) &= 
-\frac{\partial}{\partial z} \left( \frac{z - z_Q}{l^3} \right) = \left(3 
\frac{(z - z_Q)^2}{l^5} - \frac{1}{l^3} \right){.}
\end{split}
\end{equation}
%
Súčet členov na pravej strane predošlých troch rovníc je nulový, čím bolo 
dokázané, že gravitačný potenciál hmotného bodu vyhovuje Laplaceovej 
diferenciálnej rovnici~(\ref{eq:vg_laplace_cart}) pre $l > 0$.

Využitím vzťahov~(\ref{eq:l_2nd_derivatives}) a vlastností Laplaceovho 
operátora~(\ref{eq:laplace_additivity}) a (\ref{eq:laplace_homogenity}) je 
možné presvedčiť sa, že Laplaceovej rovnici vyhovuje i gravitačný potenciál 
generovaný sústavou $N$ hmotných bodov,
%
\begin{equation}
\nabla^2 \left( \sum_{i = 1}^N V_{\gidx,i} \right) = \sum_{i = 1}^N \nabla^2 
V_{\gidx,i} = 0{,}
\end{equation}
%
ako aj gravitačný potenciál všeobecného telesa v bode \emph{mimo telesa},
%
\begin{equation}
\nabla^2 V_\gidx = G\, \iiint_\tau \rho \, \left[ \frac{\partial^2}{\partial 
x^2}\left(\frac{1}{l}\right) + \frac{\partial^2}{\partial 
y^2}\left(\frac{1}{l}\right) + \frac{\partial^2}{\partial 
z^2}\left(\frac{1}{l}\right) \right] \diff\tau = 0{.}
\end{equation}

Podľa \cite{MacMillan1930} Laplace predstavil 
rovnicu~(\ref{eq:vg_laplace_cart}) po prvýkrát nie v pravouhlých súradniciach 
$x$, $y$ a $z$, ale vo sférických súradniciach,
%
\begin{equation}
\label{eq:vg_laplace_sph}
\nabla^2 V_\gidx = \frac{1}{r^2} \frac{\partial}{\partial r} \left( r^2 
\frac{\partial V_\gidx}{\partial r} \right) + \frac{1}{r^2 \, \cos\varphi} 
\frac{\partial}{\partial \varphi} \left( \cos\varphi \frac{\partial 
V_\gidx}{\partial \varphi} \right) + \frac{1}{r^2 \, 
\cos^2\varphi}\frac{\partial^2 V_\gidx}{\partial \lambda^2} = 0{,}
\end{equation}
%
kde $r$ je sprievodič, $\varphi$ je sférická šírka a $\lambda$ je sférická 
dĺžka bodu $P$ (Obrázok~\ref{fig:cart_sph}).  Existuje tiež všeobecný zápis 
Laplaceovej rovnice v ľubovoľných pravouhlých súradniciach \citep[pozri 
napríklad][]{MoritzPhysicalGeodesy}.

\begin{figure}[b]
\centering
\input{./fig-cart-sph.pdf_tex}
\caption{Pravouhlé a sférické súradnice bodu $P$}
\label{fig:cart_sph}
\end{figure}


\subsection{Harmonická funkcia}

Laplaceovej rovnici vyhovuje nielen gravitačný potenciál mimo telesa, ale 
nekonečné množstvo funkcií.  Zaveďme preto pojem \emph{harmonická funkcia}.  
Harmonická funkcia je taká funkcia, ktorá má spojité prvé a druhé parciálne 
derivácie na otvorenej množine $\Omega$ a vyhovuje Laplaceovej rovnici
%
\begin{equation}
\nabla^2 f = 0
\end{equation}
%
v každom bode oblasti $\Omega$.\footnote{Oblasť $\Omega$ môže v prípade 
gravitačného poľa predstavovať napríklad priestor mimo telesa, ktoré generuje 
gravitačné pole.}

Každá harmonická funkcia je \emph{analytická} v oblasti, v ktorej vyhovuje 
Laplaceovej rovnici \citep{MoritzPhysicalGeodesy}.  To znamená, že každá 
harmonická funkcia je spojitá, má spojité všetky derivácie a v každom bode 
oblasti $\Omega$ ju možno rozvinúť do mocninového radu (napríklad Taylorovho), 
ktorý konverguje.  Táto vlastnosť je kľúčová pre fyzikálnu geodéziu, preto sa 
pri nej na chvíľku pristavme.

Uvažujme všeobecné teleso generujúce gravitačné pole.  Nech sú dané dva body, 
$P_1(r, \varphi, \lambda)$ a $P_2(r + \Delta r, \varphi, \lambda)$, $\Delta 
r > 0$, nachádzajúce sa mimo tohto telesa a na tej istej spojnici so začiatkom 
súradnicového systému, ktorý sa nachádza v ťažisku telesa.  Vieme, že 
gravitačný potenciál je harmonická, a teda analytická, funkcia, preto ho môžeme 
v bode $P_1$ rozvinúť do Taylorovho radu, ktorý konverguje,
%
\begin{equation}
V_\gidx(P_2) = \sum_{i = 0}^\infty \frac{1}{i!} \, \frac{\partial^i 
V_\gidx(P_1)}{\partial r^i} \, \Delta r^i{.}
\end{equation}
%
Táto rovnica hovorí, že ak poznáme všetky derivácie gravitačného potenciálu 
v bode $P_1$ v smere sprievodiča $r$ a ak poznáme vzdialenosť $\Delta r$, je 
možné vypočítať (predikovať) hodnotu gravitačného potenciálu v inom bode $P_2$!  
Inými slovami, z lokálnych vlastností gravitačného potenciálu, ktoré sú 
obsiahnuté v člene $\partial^i V_\gidx \slash \partial r^i$, dokážeme určiť 
hodnotu gravitačného potenciálu v bode.

\begin{figure}[b]
\centering
\input{./fig-analytical-continuation.pdf_tex}
\caption{Analytické pokračovanie gravitačného poľa z bodu $P_1$ do bodu $P_2$}
\label{fig:analytical_continuation}
\end{figure}

% čo to je?
% súradnicový systém?
% polia
% skalár, vektor, tenzor + ukážky + kódy
% gradient, divergencia, rotácia
% Newtonov gravitačný zákon + pár slov k teórii relativity






% -----------------------------------------------------------------------------

\chapter{Gravitačné a tiažové pole homogénnej gule}

% polia, označenia, jednotky





% -----------------------------------------------------------------------------

\chapter{Gravitačný potenciál všeobecného telesa}

% rozvoj potenciálu do radu sférických harmonických funkcií






% -----------------------------------------------------------------------------

\chapter{Sférické harmonické funkcie}

% čo to je, definícia, porovnanie s jednotkovými vektormi, delenie, ukážka 
% sférických harmonických funkcií v matematickom zápise + obrázky + kód, 
% normovanie, Legendreove polynómy + funkcie, definícia, grafy + kód

\lstinputlisting[caption=Výpočet a~zobrazenie Legendreových polynómov, 
language=mypython]{../python/legendre-polynomials.py}

\begin{figure}[bt]
\centering
\includegraphics{../figs/legendre-polynomials.pdf}
\caption{Legendreove polynómy}
\end{figure}


\lstinputlisting[caption=Výpočet a~zobrazenie nenormovaných sférických 
harmonických funkcií, language=mypython]{../python/spherical-harmonics.py}

\begin{figure}[bt]
\centering
\includegraphics{../figs/spherical-harmonic-n3-m0.pdf}
\includegraphics{../figs/spherical-harmonic-n3-m1.pdf}
\includegraphics{../figs/spherical-harmonic-n3-m3.pdf}
\includegraphics{../figs/spherical-harmonic-n4-m0.pdf}
\includegraphics{../figs/spherical-harmonic-n4-m1.pdf}
\includegraphics{../figs/spherical-harmonic-n4-m4.pdf}
\caption{Nenormované plošné sférické harmonické funkcie.  Záporné hodnoty sú 
zobrazené odtieňmi modrej farby, kladné hodnoty odtieňmi červenej farby.  
\textit{Vrchný rad} (zľava): $Y_{3,0}(\varphi, \lambda)$, $Y_{3,1}(\varphi, 
\lambda)$, $Y_{3,3}(\varphi, \lambda)$.  \textit{Spodný rad} (zľava): 
$Y_{4,0}(\varphi, \lambda)$, $Y_{4,1}(\varphi, \lambda)$, $Y_{4,4}(\varphi, 
\lambda)$}
\label{fig:sh}
\end{figure}


\begin{figure}[bt]
\centering
\includegraphics{../figs/spherical-harmonic-n3-m0-3d.pdf}
\includegraphics{../figs/spherical-harmonic-n3-m1-3d.pdf}
\includegraphics{../figs/spherical-harmonic-n3-m3-3d.pdf}
\includegraphics{../figs/spherical-harmonic-n4-m0-3d.pdf}
\includegraphics{../figs/spherical-harmonic-n4-m1-3d.pdf}
\includegraphics{../figs/spherical-harmonic-n4-m4-3d.pdf}
\caption{Nenormované plošné sférické harmonické funkcie z Obr.~\ref{fig:sh} 
zobrazené ako odľahlosť nadobúdanej hodnoty od jednotkovej sféry}
\label{fig:sh3d}
\end{figure}





% -----------------------------------------------------------------------------

\section{Elipsoidické harmonické funkcie}

% čo to je, definícia, obrázky + grafy + kód, využitie






% -----------------------------------------------------------------------------

\chapter{Normálne tiažové pole}

% čo to je, prečo sa zavádza, ekvipotenciálny elipsoid, základné a odvodené 
% parametre elipsoidu, geometrické a fyzikálne parametre, základné veličiny, 
% stručne opísať normálne pole pomocou elipsoidických harmonických funkcií 
% a vysvetliť výhody takéhoto opisu (uzavreté vzťahy), somiglianov vzťah






% -----------------------------------------------------------------------------

\chapter{Poruchové pole}

% čo to je, prečo sa zavádza, využitie, základné veličiny, vzťahy medzi nimi






% -----------------------------------------------------------------------------

\chapter{Výpočet geoidu}






% -----------------------------------------------------------------------------

\section{Výpočet geoidu rozvojom do radu sférických harmonických funkcií}






% -----------------------------------------------------------------------------

\section{Výpočet geoidu astronomicko--geodetickou niveláciou}






% -----------------------------------------------------------------------------

\section{Výpočet geoidu numerickou integráciou Stokesovho a Hotine integrálov}






% -----------------------------------------------------------------------------

\chapter*{Záver}






% -----------------------------------------------------------------------------

\appendix
\chapter{Numerická aplikácia operátora gradient}
\label{app:numerical_application_of_gradient}

Uvažujme dvojrozmerný pravouhlý súradnicový systém so súradnicovými osami $x$, 
$y$.  Nech $f(x, y)$ je skalárna funkcia dvoch premenných daná vzťahom
%
\begin{equation}
\label{eq:f}
f(x, y) = \sin(2x) + \cos(2y){.}
\end{equation}
%
Aplikáciou operátora gradient na skalárnu funkciu $f(x, y)$ získame v zmysle 
rovnice~(\ref{eq:gradient}) vektorovú funkciu
%
\begin{equation}
\label{eq:gradf}
\vec f(x, y) =
\begin{bmatrix}
\dfrac{\partial f(x, y)}{\partial x} \\[2ex]
\dfrac{\partial f(x, y)}{\partial y}
\end{bmatrix}
=
\begin{bmatrix}
2 \cos(2x) \\[2ex]
-2 \sin(2y)
\end{bmatrix}
{.}
\end{equation}

Ukážka numerického výpočtu funkcie~(\ref{eq:f}) a jej 
gradientu~(\ref{eq:gradf}) na~intervale $x, y \in [-1, 1]$ je uvedená 
v Zdrojovom~kóde~\ref{src:f_gradf}.  Grafické znázornenie je uvedené na 
Obr.~\ref{fig:f_gradf}.  Všimnime si, v ktorých bodoch má šípka malú či veľkú 
dĺžku a pokúsme sa danú situáciu interpretovať.

\lstinputlisting[caption=Výpočet a~zobrazenie funkcie dvoch premenných a~jej 
gradientu v programovacom jazyku Python, language=mypython, label=src:f_gradf, 
captionpos=t]{../python/f-gradf.py}

\begin{figure}[bt]
\centering
\includegraphics{../figs/f-gradf.pdf}
\caption{Skalárna funkcia $f(x, y) = \sin(2x) + \cos(2y)$ a~jej gradient 
$\nabla f(x, y)$ na~intervale $x, y \in [-1, 1]$.  Skalárna funkcia je 
znázornená hypsometricky.  Gradient je zobrazený orientovanými šípkami, pričom 
šípka udáva smer a veľkosť najväčšieho nárastu funkcie $f(x, y)$ v danom bode}
\label{fig:f_gradf}
\end{figure}

% Bibliography
% -----------------------------------------------------------------------------

\bibliographystyle{apalike}
\bibliography{references.bib}

\end{document}

% =============================================================================
% End of the code

